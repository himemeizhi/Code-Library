\subsection{other}
\subsubsection{Pick's theorem}
给定顶点座标均是整点(或正方形格点)的简单多边形\\
A:面积\\
i:内部格点数目\\
b:边上格点数目\\
$A=i+\frac{b}{2}-1$\\

取格点的组成图形的面积为一单位。在平行四边形格点,皮克定理依然成立。套用于任意三角形格点,皮克定理则是\\
$A=2\times i+b-2$\\

\subsubsection{Triangle}
Area:\\
$p=\frac{a+b+c}{2}$\\
$area=\sqrt{p\times (p-a)\times (p-b)\times (p-c)}$\\
$area=\frac{a\times b\times \sin(\angle C)}{2}$\\
$area=\frac{a^2\times \sin(\angle B)\times \sin(\angle C)}{2\times \sin(\angle B+\angle C)}$\\
$area=\frac{a^2}{2\times (\cot(\angle B)+\cot(\angle C))}$\\
\\
centroid:\\
    center of mass\\
    intersection of triangle's three triangle medians\\
\\
Trigonometric conditions:\\
$\tan{\frac{\alpha}{2}}\tan{\frac{\beta}{2}}+\tan{\frac{\beta}{2}}\tan{\frac{\gamma}{2}}+\tan{\frac{\gamma}{2}}\tan{\frac{\alpha}{2}}=1$\\
$\sin^2{\frac{\alpha}{2}}+\sin^2{\frac{\beta}{2}}+\sin^2{\frac{\gamma}{2}}+2\sin{\frac{\alpha}{2}}\sin{\frac{\beta}{2}}\sin{\frac{\gamma}{2}}=1$\\
\\
Circumscribed circle:\\
$diameter = \frac{abc}{2\cdot\text{area}} = \frac{|AB| |BC| |CA|}{2|\Delta ABC|} \\\\
         = \frac{abc}{2\sqrt{s(s-a)(s-b)(s-c)}}\\\\
         = \frac{2abc}{\sqrt{(a+b+c)(-a+b+c)(a-b+c)(a+b-c)}}$\\
$diameter=\sqrt{\frac{2 \cdot \text{area}}{\sin A \sin B \sin C}}$\\
$diameter=\frac{a}{\sin A}=\frac{b}{\sin B}=\frac{c}{\sin C}$\\
\\
Incircle:\\
$inradius=\frac{2\times area}{a+b+c}$\\
coordinates(x,y)=$\bigg(\frac{a x_a+b x_b+c x_c}{a+b+c},\frac{a y_a+b y_b+c y_c}{a+b+c}\bigg) = \frac{a}{a+b+c}(x_a,y_a)+\frac{b}{a+b+c}(x_b,y_b)+\frac{c}{a+b+c}(x_c,y_c)$\\
\\
Excircles:\\
radius[a]=$\frac{2\times area}{b+c-a}$\\
radius[b]=$\frac{2\times area}{a+c-b}$\\
radius[c]=$\frac{2\times area}{a+b-c}$\\
\\
Steiner circumellipse (least area circumscribed ellipse)\\
    area=$\Delta \times \frac{4\pi}{3\sqrt{3}}$\\
    center is the triangle's centroid.\\
\\
Steiner inellipse ( maximum area inellipse )\\
    area=$\Delta \times \frac{\pi}{3\sqrt{3}}$\\
    center is the triangle's centroid.\\
\\
Fermat Point:
\begin{enumerate}
\item 当有一个内角不小于120°时,费马点为此角对应顶点。
\item 当三角形的内角都小于120°
\begin{enumerate}
\item 以三角形的每一边为底边,向外做三个正三角形$\Delta$ABC',$\Delta$BCA',$\Delta$CAB'。
\item 连接CC'、BB'、AA',则三条线段的交点就是所求的点。
\end{enumerate}
\end{enumerate}

\subsubsection{Ellipse}
$\frac{(x-h)^2}{a^2} + \frac{(y-k)^2}{b^2} = 1$\\

$x=h+a \times \cos(t)$\\
$y=k+b \times \sin(t)$\\

area=$\pi \times a \times b$\\
distance from center to focus: $f=\sqrt{a^2-b^2}$\\
eccentricity: $e=\sqrt{a-\frac{b}{a}^2}=\frac{f}{a}$\\
focal parameter: $\frac{b^2}{\sqrt{a^2-b^2}}=\frac{b^2}{f}$\\

\begin{lstlisting}[language=C++]
inline double circumference(double a,double b) // accuracy: pow(0.5,53);
{
    static double digits=53;
    static double tol=sqrt(pow(0.5,digits));
    double x=a;
    double y=b;
    if(x<y)
        std::swap(x,y);
    if(digits*y<tol*x)
        return 4*x;
    double s=0,m=1;
    while(x>(tol+1)*y)
    {
        double tx=x;
        double ty=y;
        x=0.5f*(tx+ty);
        y=sqrt(tx*ty);
        m*=2;
        s+=m*pow(x-y,2);
    }
    return pi*(pow(a+b,2)-s)/(x+y);
}
\end{lstlisting}

\subsubsection{about double}

如果sqrt(a), asin(a), acos(a) 中的a是你自己算出来并传进来的,那就得小心了。如果a本来应该是0的,由于浮点误差,可能实际是一个绝对值很小的负数(比如$-1^{-12}$),这样sqrt(a)应得0的,直接因a不在定义域而出错。类似地,如果a本来应该是$\pm$1,则asin(a)、acos(a)也有可能出错。因此,对于此种函数,必需事先对a进行校正。\\

现在考虑一种情况,题目要求输出保留两位小数。有个case的正确答案的精确值是0.005,按理应该输出0.01,但你的结果可能是0.005000000001(恭喜),也有可能是0.004999999999(悲剧),如果按照printf("\%.2lf", a)输出,那你的遭遇将和括号里的字相同。\\
如果a为正,则输出a + eps, 否则输出a - eps。\\

不要输出-0.000\\

注意double的数据范围\\

\begin{tabular}{|l|l|}
\hline
$a=b$ & fabs(a-b)<eps\\
\hline
$a\neq b$ & fabs(a-b)>eps\\
\hline
$a<b$ &  a+eps<b\\
\hline
$a\leq b$ & a<b+eps\\
\hline
$a>b$  & a>b+eps\\
\hline
$a\geq b$ & a+eps>b\\
\hline
\end{tabular}

\subsubsection{trigonometric functions}

\begin{tabular}{|l|l|l|}
\hline
& input & output\\
\hline
sin & radian & $[-1,+1]$\\
\hline
cos & radian & $[-1,+1]$\\
\hline
tan & radian & $(-\infty,+\infty)$\\
\hline
asin & $[-1,+1]$ & $[-\frac{\pi}{2},+\frac{\pi}{2}]$\\
\hline
acos & $[-1,+1]$ & [0,$\pi$]\\
\hline
atan & $(-\infty,\infty)$ & $[-\frac{\pi}{2},+\frac{\pi}{2}]$\\
\hline
atan2 & (y,x) & $\tan(\frac{y}{x}) \in [-\pi,+\pi]$ (watch out if x=y=0)\\
\hline
\end{tabular}

\begin{tabular}{|l|l|}
\hline
exp & $x^e$\\
\hline
log & ln\\
\hline
log10 & $log_{10}$\\
\hline
ceil & smallest interger $\geq$ x (watch out x<0\\
\hline
floor & greatest interger $\leq$ x (watch out x<0\\
\hline
trunc & nearest integral value close to 0\\
\hline
nearybyint & round to intergral, up to fegetround\\
\hline
round & round with halfway cases rounded away from zero\\
\hline
\end{tabular}

\subsubsection{round}
\begin{enumerate}
\item cpp:  四舍六入五留双
\begin{enumerate}
\item 当尾数小于或等于4时,直接将尾数舍去
\item 当尾数大于或等于6时,将尾数舍去并向前一位进位
\item 当尾数为5,而尾数后面的数字均为0时,应看尾数“5”的前一位:若前一位数字此时为奇数,就应向前进一位;若前一位数字此时为偶数,则应将尾数舍去。数字“0”在此时应被视为偶数
\item 当尾数为5,而尾数“5”的后面还有任何不是0的数字时,无论前一位在此时为奇数还是偶数,也无论“5”后面不为0的数字在哪一位上,都应向前进一位
\end{enumerate}
\item java: add 0.5,then floor
\end{enumerate}

\subsubsection{rotation matrix}
original matrix:\\
$\begin{bmatrix}
x\\
y
\end{bmatrix}$

$\begin{bmatrix}
\cos(\theta) & -\sin(\theta) \\
\sin(\theta) & \cos(\theta)
\end{bmatrix}$

3-dimension:\\
$\begin{bmatrix}
x\\
y\\
z
\end{bmatrix}$

$R_x(\theta) = \begin{bmatrix}
1 & 0 & 0 \\
      0 & \cos \theta & - \sin \theta \\
      0 & \sin \theta  &  \cos \theta \\
      \end{bmatrix} \\
      R_y(\theta) = \begin{bmatrix}
      \cos \theta & 0 & \sin \theta \\
          0 & 1 & 0 \\
          - \sin \theta & 0 & \cos \theta \\
          \end{bmatrix} \\
          R_z(\theta) = \begin{bmatrix}
          \cos \theta & - \sin \theta & 0 \\
              \sin \theta & \cos \theta & 0\\
              0 & 0 & 1\\
              \end{bmatrix}$\\
              rotation by unit vector ${v} = (x,y,z)$:\\
              $\begin{bmatrix}
              \cos \theta + (1 - \cos \theta) x^2
              & (1 - \cos \theta) x y - (\sin \theta) z 
              & (1 - \cos \theta) x z + (\sin \theta) y  
              \\
                  (1 - \cos \theta) y x + (\sin \theta) z 
                  & \cos \theta + (1 - \cos \theta) y^2
                  & (1 - \cos \theta) y z - (\sin \theta) x
                  \\
                      (1 - \cos \theta) z x - (\sin \theta) y
                      & (1 - \cos \theta) z y + (\sin \theta) x
                      & \cos \theta + (1 - \cos \theta) z^2 
                      \end{bmatrix}$ 
                      we use transform matrix muliply our original matrix\\

and we can presetation a transformation as a $4\times 4$ matrix:\\
$\begin{bmatrix}
a_{11} & a_{12} & a_{12} & a_{14}\\
a_{21} & a_{22} & a_{22} & a_{24}\\
a_{31} & a_{32} & a_{32} & a_{34}\\
a_{41} & a_{42} & a_{42} & a_{44}
\end{bmatrix}$\\
Matrix $\begin{bmatrix}
a_{11} & a_{12} & a_{12}\\
a_{21} & a_{22} & a_{22}\\
a_{31} & a_{32} & a_{32}
\end{bmatrix}$ presetation the transformation as same as $3\times 3$ matrx.\\
Matrix $\begin{bmatrix}a_{14} \\ a_{24} \\ a_{34} \end{bmatrix}$ as translation.\\
Matrix $\begin{bmatrix}a_{41} & a_{42} & a_{43} \end{bmatrix}$ as projection.\\
Matrix $\begin{bmatrix}a_{44}\end{bmatrix}$ as scale.\\

original Matrix:\\
$\begin{bmatrix}x\\y\\z\\Scale\end{bmatrix}$\\
